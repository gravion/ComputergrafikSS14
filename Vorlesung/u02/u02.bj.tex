\documentclass[11pt]{article}

\usepackage[utf8]{inputenc}
\usepackage{geometry}
\usepackage{color,graphicx,framed}
\usepackage{amsmath,amsfonts,amssymb}
\usepackage{listings}
\usepackage[section]{placeins}
\usepackage{pstricks,pst-tree,pst-node}
\usepackage{dsfont}
\usepackage{ulem}
\usepackage{tikz}
\usetikzlibrary{shapes,arrows,positioning,calc}

\title{Computergraphik SS14\\Übungsblatt 02}
\author{Björn Rathjen}
\date{zu 02.05.14}

\begin{document}
\maketitle
\newpage

\section*{10. Homogene Koordinaten (10 Punkte)}
Dies ist eine alte Klausuraufgabe.
\begin{enumerate}
\item[(a)] Stellen Sie für die Gerade durch die Punkte (2, 3) und (4, 5) in der Ebene eine Geradengleichung der Form
\begin{eqnarray*}
ax + by + cw &=& 0 
\end{eqnarray*}
in homogenen Koordinaten (x, y, w) auf.
\begin{eqnarray*}
\begin{pmatrix}
2 \\ 3\\
\end{pmatrix}
-
\begin{pmatrix}
4 \\ 5 \\
\end{pmatrix}
&=&
\begin{pmatrix}
-2 \\ -2 \\
\end{pmatrix}\\
g : \begin{pmatrix}
2 \\ 3 \\
\end{pmatrix}
+
\lambda \cdot \begin{pmatrix}
-2 \\ -2 \\
\end{pmatrix} &=& \begin{pmatrix}
x \\ y \\
\end{pmatrix}\\
\vec{a} \times \vec{b} &=& \begin{pmatrix} 2\\ 3\\ 1\\
\end{pmatrix} \times \begin{pmatrix}
4 \\ 5 \\ 1 \\
\end{pmatrix}\\
&=& \begin{pmatrix}
3\cdot 1 - 1\cdot 5 \\
1\cdot 4 - 2\cdot 1 \\
2\cdot 5 - 3\cdot 4 \\
\end{pmatrix} \\
&=&
\begin{pmatrix}
-2 \\ 2 \\ -2\\
\end{pmatrix} \\
&\Rightarrow & -2x + 2y -2w = 0 \\
& \Leftrightarrow & -x +y -w = 0
\end{eqnarray*}
\item[(b)] Die Gleichung
$$x^2 + 2xw + y^2 - 12w^2 - 3wy = 0$$ 
in homogenen Koordinaten (x, y, w) beschreibt einen Kreis in der Ebene. Bestimmen Sie seinen Radius und den Mittelpunkt (in kartesischen Koordinaten).
$$ (x+w)^2 + (y-1,5w)^2 - 15,25w^2 = 0 $$
Aus dieser Kreisgleichung lassen sich die Koordinaten und der Radius ablesen. Da karthesische Koordinaten ist $w=1$ somit ist der Mittelpunkt und Radus :
$$ P_{Mittelpunkt}(-1|1,5)$$
$$ r = \sqrt{15,25}$$
\end{enumerate}
\section*{11. Zentralprojektion (10 Punkte)}
Bestimmen Sie die Abbildungsmatrix A (in homogenen Koordinaten) für die Zentralprojektion vom Punkt P = (4, 2) auf die Gerade g: $2x + y + 1 = 0$.
\begin{eqnarray}
(P \times Q) \times g &=& 
\left(
\begin{pmatrix}
4 \\ 2 \\ 1
\end{pmatrix} \times
\begin{pmatrix}
x \\ y \\ 1 \\
\end{pmatrix} \right) \times
\begin{pmatrix}
2 \\ 1 \\ 1 \\
\end{pmatrix} \\
&=& \begin{pmatrix}
2-y \\ x-4 \\ 4y-2x \\
\end{pmatrix}\times 
\begin{pmatrix}
2 \\ 1 \\ 1 \\
\end{pmatrix} \\
&=& \begin{pmatrix}
x-4 - 4y+2x \\
(4y-2x) \cdot 2 - (2-y) \\
(2-y) - (x-4)\cdot 2
\end{pmatrix} \\
&=&
\begin{pmatrix}
3x -4y -4 \\
-4x +5y -2 \\
-2x -y + 10 \\ 
\end{pmatrix}\\
&\Rightarrow & A= \begin{pmatrix}
3 & -4 & -4 \\
-4 & 5 & -2 \\
-2 & -1 & 10 \\
\end{pmatrix} 
\end{eqnarray}
\section*{12. Koordinatensysteme (12 Punkte, zu bearbeiten bis Donnerstag 7. Mai 2014)}
\begin{enumerate}
\item[(a)] Eine Kamera steht im Punkt
$\begin{pmatrix}
4 \\
5 \\
3 \\
\end{pmatrix}$
und blickt in Richtung auf den Punkt
$\begin{pmatrix}
7 \\
5 \\
4 \\
\end{pmatrix}$
Bestimmen Sie das entsprechende rechtwinklige Augenkoordinatensystem so, dass
die Kamera aufrecht steht.
\item[(b)] Bestimmen Sie die $4 \times 4$ - Transformationsmatrix zur Umrechnung von Weltkoordinaten in Augenkoordinaten. Sie sollen in der Lage sein, die Aufgabe auch mit abgeänderten Zahlen zu lösen.
\end{enumerate}
\section*{13. Scherungen in der Ebene (8 Punkte)}
Die Transformationsmatrizen
$\begin{pmatrix}
1 & a & 0 \\
0 & 1 & 0 \\
0 & 0 & 1 \\ 
\end{pmatrix}$
und
$\begin{pmatrix}
1 & 0 & 0 \\
b & 1 & 0 \\
0 & 0 & 1 \\
\end{pmatrix}$
beschreiben eine Scherung in x-Richtung bzw. in y-Richtung $(a, b \in \mathbb{R})$.
\begin{enumerate}
\item[(a)] Welche Punkte der Ebene werden dabei in sich selbst überführt (Fixpunkte)?\\
Bei Scherung in x-Richtung alle Punkte auf der x-Achse, bei Scherung in y-Richtung alle auf der y-Achse.
Weil bei 
\begin{description}
\item[$x=0$]
\begin{eqnarray*}
\begin{pmatrix}
1 & 0 & 0 \\
b & 1 & 0 \\
0 & 0 & 1 \\ 
\end{pmatrix}
\times
\begin{pmatrix}
0 \\ y \\ 1
\end{pmatrix}
&=& 
\begin{pmatrix}
0 \\ y \\ 1
\end{pmatrix}
\end{eqnarray*}
\item[$y=0$]
\begin{eqnarray*}
\begin{pmatrix}
1 & a & 0 \\
0 & 1 & 0 \\
0 & 0 & 1 \\ 
\end{pmatrix}
\times
\begin{pmatrix}
x \\ 0 \\ 1
\end{pmatrix}
&=& 
\begin{pmatrix}
x \\ 0 \\ 1
\end{pmatrix}
\end{eqnarray*}
\end{description}
\item[(b)] Welche Geraden werden in sich selbst überführt (Fixgeraden)?\\
Bei Scherung in x-Richtung alle Geraden die parallel zur x-Achse, bei Scherung in y-Richtung alle Geraden parallel zur y-Achse.
\begin{description}
\item[in y-Richtung]
$$
\begin{pmatrix}
1 & 0 & 0 \\
b & 1 & 0 \\
0 & 0 & 1 \\ 
\end{pmatrix}
\times
\begin{pmatrix}
x \\ y \\ 1
\end{pmatrix}
= 
\begin{pmatrix}
x \\ y + by \\ 1
\end{pmatrix}
$$
\item[in x-Richtung]
$$
\begin{pmatrix}
1 & a & 0 \\
0 & 1 & 0 \\
0 & 0 & 1 \\ 
\end{pmatrix}
\times
\begin{pmatrix}
x \\ y \\ 1
\end{pmatrix}
=
\begin{pmatrix}
x + ay \\ y \\ 1
\end{pmatrix}$$
\end{description}
\item[(c)] Wenden Sie eine Scherung mit $a = 0,7$ und unabhängig davon eine Scherung mit
$b = -0,3$ auf folgende Abbildung an, und zeichnen Sie die Ergebnisse.
{\color{red}Zeichnung}

\item[(d) (0 Punkte)] Scherungen erhalten den Flächeninhalt.
\item[(e) (0 Punkte)] Rotationen erhalten ebenfalls den Flächeninhalt. Jede Rotation kann als Produkt von drei Scherungen geschrieben werden.
\end{enumerate}
\end{document}