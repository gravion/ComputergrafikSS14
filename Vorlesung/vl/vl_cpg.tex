\documentclass[11pt]{article}

\usepackage[utf8]{inputenc}
\usepackage{geometry}
\usepackage{color,graphicx,framed}
\usepackage{amsmath,amsfonts,amssymb,MnSymbol}
\usepackage{listings}
\usepackage[section]{placeins}
\usepackage{pstricks,pst-tree,pst-node}
\usepackage{dsfont}
\usepackage{ulem}

\usepackage{program}

\title{Computergraphik\\Vorlesung}
\author{Björn Rathjen}
\date{SS14}
\begin{document}
\maketitle
\newpage\tableofcontents

\section{Vorlesung 23.04.14}
\subsection{Die projektive Ebene}
\subsection{homogene Koordinaten}
\begin{figure}[!hb]
\begin{minipage}[c]{8cm}
$$ 
\begin{pmatrix}
x \\ y \\ z \\
\end{pmatrix},
\begin{pmatrix}
2 \\ 4 \\ 6
\end{pmatrix},
\begin{pmatrix}
6 \\ 12 \\ 18 
\end{pmatrix},
\begin{pmatrix}
-1 \\ -2 \\ -3 
\end{pmatrix} =
\begin{pmatrix}
\frac{1}{3}\\ \frac{2}{3} \\ 1 \\
\end{pmatrix}
$$
\end{minipage}
\begin{minipage}[c]{6cm}
stellen den selben Punkt in homogenen Koordinaten dar.
\end{minipage}
\end{figure}
vergleiche Geradengleichung :
\begin{figure}[!hb]
\begin{minipage}[c]{6cm}
\begin{eqnarray*}
ax + by + c & = & 0 \\
3x + 4y + 5 & = & 0 \\
6x + 8y 	b 10 & = & 0 \\
\end{eqnarray*}
\end{minipage}
\begin{minipage}[c]{6cm}
\begin{minipage}[r]{3cm}
\[
\left.
\begin{array}{c}
\left( 3,4,5 \right) \\
\left( 6,8,10 \right) \\
\end{array}
\right\rbrace
\]
\end{minipage}\hfill
\begin{minipage}[r]{3cm}
stellen die selben Geradengleichung dar
\end{minipage}
\end{minipage}
\end{figure}
Die Punkte in der projektiven Ebene sind die Äquivalenzklassen von Vektoren $ \begin{pmatrix}
x \\ y \\ w
\end{pmatrix} \ in \mathbb{R}^3 $mit $\begin{pmatrix}
x \\ y \\ w 
\end{pmatrix} \neq \begin{pmatrix}
0 \\ 0 \\ 0 \\
\end{pmatrix}$ wobei $\vec{a}$ und $\vec{b}$ als äquivalent betrachtet werden, wenn : $\exists \lambda \neq 0 : \vec{a} = \lambda \vec{b}$ 
\subsection{das räumliche Modell der projektiven Ebene}
Die Punkte sind Geraden durch den Ursprung in $\mathbb{R}^3$ (Äquivalenzklassen, bis auf den Ursprung $\begin{pmatrix}
0 \\ 0 \\ 0 \\
\end{pmatrix}$
{\color{red} Zeichnung Koordinatandsystem Ebene}
%\caption{•}
Die Punkte mit $w = 0$ heißen \underline{Fernpunkte}. Ihnen entsprechen kein Punkt mit karthesischen Koordinaten.
\newline
\newline

Die Punkte der \underline{euklidischen Ebene} sind alle Punkte außer den Fernpunkten.

Geraden der projektiven Ebene bestehen aus allen Punkten $\begin{pmatrix}
x \\ y \\ w 
\end{pmatrix}$, die eine Geradengleichung 
$$ ax + by + c = 0 $$
mit den Koeffizienten $(a,b,c) \neq (0,0,0)$ \underline{erfüllen}.\\
Für $w=1$ :
$$ ax + by + c = 0 $$

Geradengleichung : 
$\begin{pmatrix}
a \\ b \\ c
\end{pmatrix}
\cdot 
\begin{pmatrix}
x \\ y \\ w 
\end{pmatrix}
= 0 $

In der projektiven Ebene
\begin{enumerate}
\item[a)] geht durch je zwei Punkte genau eine Gerade
\item[b)] schneiden sich je zwei Geraden in genau einem Punkt
\end{enumerate}

Beweis zu b):
$$ ax + by + c = 0$$
$$ dx + ey + fw = 0$$
homogenes Gleichungssystem in x, y, w \\
Matrix $\begin{pmatrix}
a & b & c \\
d & e & f \\
\end{pmatrix}$ rg = 0 ist ausgeschlossen\\
Wenn der Rang = 1 wäre müsste $(a,b,c) = \lambda(d,e,f)$ sein, dann wären es zwei gleiche Geraden\\
$\Rightarrow $ Rang = 2 $\Rightarrow $ die Lösung ist eine \underline{Gerade} durch den Ursprung. (ein projektiver Punkt)\\
\newline

Beweis zu a):
$\begin{array}{cc}
ax + by + cw &  = 0 \\ 
au + bv + fz &= 0 \\ 
\end{array} \text{ mit }
\begin{pmatrix}
x \\ y \\ w \\
\end{pmatrix} \neq \lambda \begin{pmatrix}
u \\ v \\ z 
\end{pmatrix}
\forall \lambda $
Gleichungssystem in (a,b,c) $\begin{pmatrix}
x & y & w \\
u & v & z \\
\end{pmatrix}$\\
Berechnung geht am besten mit dem Kreuzprodukt :
$$ \begin{pmatrix}
x_1 \\ x_2 \\ x_3 \\
\end{pmatrix}
\times
\begin{pmatrix}
y_1 \\ y_2 \\ y_3 \\
\end{pmatrix}
=
\begin{pmatrix}
\begin{vmatrix}
x_2 & y_2 \\
x_3 & y_3 \\ 
\end{vmatrix}\\
\begin{vmatrix}
x_3 & y_3 \\
x_1 & y_1 \\ 
\end{vmatrix}\\
\begin{vmatrix}
x_1 & y_1 \\
x_2 & y_2 \\ 
\end{vmatrix}
\end{pmatrix} =
\begin{pmatrix}
x_2y_3 - x_3y_2 \\
x_3y_1 - x_1y_3 \\
x_1y_2 - x_2y_1 \\
\end{pmatrix}
$$
liefert einen Vektor der senkrecht auf 
$ \begin{pmatrix}
x_1 \\ x_2 \\ x_3 \\
\end{pmatrix}
$ und $
\begin{pmatrix}
y_1 \\ y_2 \\ y_3 \\
\end{pmatrix}$ steht.\\
Punkt $\begin{pmatrix}
x \\ y\\ w \\
\end{pmatrix}$ liegt auf Gerade 
$\begin{pmatrix}
a \\ b\\ c \\
\end{pmatrix}$ (Inzidenz)
$$ \Leftrightarrow 
\begin{pmatrix}
x \\ y\\ w \\
\end{pmatrix}
\cdot
\begin{pmatrix}
a \\ b\\ c \\
\end{pmatrix} \neq 0 \Leftrightarrow 
\begin{pmatrix}
x \\ y\\ w \\
\end{pmatrix}
\perp
\begin{pmatrix}
a \\ b\\ c \\
\end{pmatrix}
$$
Konstruktion
\begin{enumerate}
\item[a)] Gerade durch zwei Punkte
\item[b)] Punkt auf zwei Geraden
\end{enumerate}
gesucht ist ein Vektor, der auf zwei anderen senkrecht steht\\
$\Rightarrow $ Kreuzprodukt (keine Dimension notwendig, Rechnung ohne Fallunterscheidung\\

parallele Geraden schneiden sich in einem Fernpunkt.\\

Ausser den Geraden der Euklidischen Ebene enthält die projektive Ebene noch eine zusätzliche Gerade:\\
die Ferngerade, die alle Fernpunkte enthält.
$$ ax + by + cw = 0 \text{ mit }w=0 \; \; \;
\begin{pmatrix}
a \\ b \\ c
\end{pmatrix}
=
\begin{pmatrix}
a \\ b \\ 0
\end{pmatrix}
$$
\subsection{Kugelmodell}
räumliches Modell mit der Einheitssphäre schneiden:
\begin{figure}[!hb]
\begin{minipage}[l]{6cm}
\begin{tabular}{ccc}
Punkte & $\equiv$ & Paare gegenüberliegender Geraden \\
Geraden & $\equiv$ & Großkreise
\end{tabular}
\end{minipage}
\hfill
\begin{minipage}[c]{8cm}
{\color{red}Zeichnung der Kugel}
\end{minipage}
\end{figure}

\begin{description}
\item[Dualität] Punkte und Geraden werden in der projektiven Geometrie völlig gleich behandelt. Man kann jederzeit \underline{überall} Punkt durch eine Gerade ersetzen.
\end{description}
{\color{red}Zeichnung Satz von Pappus}
Projektion : (im Raum)\\
\begin{figure}[!hb]
\begin{minipage}[l]{4cm}
\begin{tabular}{cr}
Projektionszentrum &  A \\
Projektionsebene & $\pi$
\end{tabular}
\end{minipage}
\hfill
\begin{minipage}[c]{8cm}
{\color{red}Zeichnung punkt auf ebene projeziert}
\end{minipage}
\end{figure}

Projektion ist eine Abbildung von $\mathbb{R}^3 - \text{(durch)}\{A\}\rightarrow \pi$

\begin{enumerate}
\item[1.)] konstruiere die Gerade durch A und $x \in \mathbb{R}^3 \notin \{A\}$
\item[2.)] Schneide die Gerade mit $\pi$
\end{enumerate}
{\color{red}Zeichnung Baum}
Spezialfall :\\
\begin{figure}[!hb]
\begin{minipage}[c]{5cm}
{\color{red}Zeichnung Baum 2}
\end{minipage}
\hfill
\begin{minipage}[l]{7cm}
Beispiel: Schatten eines Baumes
{\color{green}eventuell als caption}
\end{minipage}
\end{figure}
\subsection{Der projektive Raum}
analog nur mit einer Koordinate mehr
$$
\begin{pmatrix}
x\\ y \\ z \\ w
\end{pmatrix} \in \mathbb{R}^4 \text{ Punkte}
$$
Dualität zwischen Punkten und Ebenen
\subsubsection{Projektionen in der Ebene}
{\color{red}Zeichnung Projektionsgerade } A(1/0)\\
Punkte $\begin{pmatrix}
x \\ y \\ w
\end{pmatrix}$ projezieren\\
mit A verbinden $A = \begin{pmatrix} 1 \\ 0 \\ 1 \end{pmatrix}$
$$
\begin{pmatrix} 1 \\ 0 \\ 1 \end{pmatrix}
\times
\begin{pmatrix} x \\ y \\ z \end{pmatrix}
=
\begin{pmatrix} -y \\ x-w \\ y \end{pmatrix}
 \text{ $\leftarrow $ Verbindungsgerade} $$
Schneiden mit der Geraden 
$$ x-y = 0 \;\; 
\begin{pmatrix} -y \\ x-w \\ y \end{pmatrix}
\times
\begin{pmatrix} 1 \\ -1 \\ 0 \end{pmatrix}
=
\begin{pmatrix} y \\ y \\ y-x+w \end{pmatrix}
$$
\subsubsection{Projektionsabbildung}
$$
\begin{pmatrix} x \\ y \\ z \end{pmatrix}
\rightmapsto
\begin{pmatrix} y \\ y \\ y-x+w \end{pmatrix}
=
\underset{\begin{minipage}[c]{3cm}
Abbildungsmatrix dieser projektiven Abbildung
\end{minipage}}{\underbrace{\begin{pmatrix}
0 & 1 & 0 \\
0 & 1 & 0 \\
1 & 1 & 1 \\
\end{pmatrix}}}
\begin{pmatrix} x \\ y \\ z \end{pmatrix}
$$
In karthesischen Koordinaten
$$ 
\begin{pmatrix}
x \\ y
\end{pmatrix}
\mapsto
\begin{pmatrix}
\frac{y}{y-x+1} \\ \frac{y}{y-x+1}
\end{pmatrix}
\; \; \text{ Fernpunkt bei $y-x+1 = 0$}
$$
\section{Vorlesung 30.04.14}
\subsection{Die klassische Rendering-Pipeline\\andere Organisationen sind möglich}
{\color{red} Zeichnung der Stationen der Pipeline}
\FloatBarrier
\begin{figure}[!hb]
\begin{minipage}[c]{5cm}
{\color{red}Koordinatensystem und Augenkoordinatensystem} 
\end{minipage}
\hfill
\begin{minipage}[r]{6cm}
$\vec{n}$ gegeben \\
in der Regel möchte man v ("oben")\\
möglichst senkrecht\\
\begin{eqnarray}
\vec{u} &=& -\vec{n} \times \begin{pmatrix}
0 \\ 0 \\ 1
\end{pmatrix}\\
\vec{v} = \vec{u} \times -\vec{n}\\
\end{eqnarray}
\end{minipage}
\end{figure}
{\color{red} Zeichnung Sichtbarkeitsstrumpf und Haltepunkt}
\subsection{normalisierte Gerätekoordinaten (NDC)}

\begin{figure}[!hb]
\begin{minipage}[l]{5cm}
{\color{red} Würfelzeichnung}
\end{minipage}
\hfill
\begin{minipage}[l]{5cm}
\begin{eqnarray*}
\text{Strumpf} &\rightarrow & \text{"Einheitswürfel"}\\
& & \left[-1,1\right] \times \left[-1,1\right] \times \left[-1,1\right]\\
\text{Augpunkt} &\rightarrow & \text{Fernpunkt in z-Richtung}\\
\end{eqnarray*}
\end{minipage}
\end{figure}
Projektion durch ganz einfach Parallelprojektion durch weglassen dar z-Koordinaton\\
Tiefeninformationen (Welches Opjekt liegt vor einem anderen ?) ist an der z-Koordinaten abzulesen (das Opjekt mit den kleinsten z-Koordinaten)
\subsubsection{Spezialfall :}
\begin{figure}[!hb]
\begin{minipage}[l]{4cm}
{\color{red} Zeichnung Öffnungswinkel}
\end{minipage}
\hfill
\begin{minipage}[c]{6.5cm}
Transformation zum Spezialfall
\begin{enumerate}
\item[1)] Alles mit $\frac{1}{\text{nah}}$ skalieren 
$\begin{pmatrix}
\frac{1}{\text{nah}} & 0 & 0 & 0 \\
0 & \frac{1}{\text{nah}} & 0 & 0 \\
0 & 0 & \frac{1}{\text{nah}} & 0 \\
0 & 0 & 0 & \frac{1}{\text{nah}} \\
\end{pmatrix}$
\item[2)] Bildrechteck auf $2y\times 2$-Rechteck skalieren
$\begin{pmatrix}
\frac{2}{\text{r' - l'}} & 0 & 0 & 0 \\
0 & \frac{2}{\text{t' - b'}} & 0 & 0 \\
0 & 0 & 1 & 0 \\
0 & 0 & 0 & 1 \\
\end{pmatrix}$
\item[3)] Scherung in u und v-Richtung um das Rechteck zu zentrieren. Nun ist r" - l" = 2 und t" - b" = 2 $\rightarrow$ nach Schritt 1 und 2 \\
\end{enumerate}
\end{minipage}
\end{figure}
bezogen auf $*^1$ und wenn Zentrum nicht gleich $\begin{pmatrix}
0 & 0 & 0 
\end{pmatrix}$
$$ \begin{pmatrix}
x \\ y \\ n \\ 
\end{pmatrix}
\rightarrow
\begin{pmatrix}
x - \frac{\text{l" + r"}}{2}n \\
y - \frac{\text{l" + r"}}{2}n \\
n
\end{pmatrix}$$
\subsection{Scherungsmatrix}
$$\begin{pmatrix}
1 & 0 & \frac{\text{l" + r"}}{2} & 0 \\
0 & 1 & \frac{\text{l" + r"}}{2} & 0 \\
0 & 0 & 1 & 0 \\
0 & 0 & 0 & 1 \\
\end{pmatrix} $$
\begin{figure}[!hb]
\begin{minipage}[l]{5cm}
{\color{red}Zeichnung u,v Koordinaten}
\end{minipage}
\hfill
\begin{minipage}[c]{6cm}
\begin{eqnarray*}
\text{nah} & & \\
\begin{pmatrix}
\pm 1 \\ \pm 1 \\ -1 
\end{pmatrix}
 \rightmapsto 
\begin{pmatrix}
\pm 1 \\ \pm 1 \\ -1 
\end{pmatrix}
& &
M 
\begin{pmatrix}
\pm 1 \\ \pm 1 \\ -1 \\ 1 \\
\end{pmatrix}
= 
\begin{pmatrix}
\pm 1 \\ \pm 1 \\ -1 \\ 1 \\
\end{pmatrix} \lambda * 1 \text{  } \,\, \lambda =1\\
\text{fern} & & \text{wähle } \mu = f\\
\begin{pmatrix}
\pm f \\ \pm f \\ -f 
\end{pmatrix}
\rightmapsto 
\begin{pmatrix}
\pm 1 \\ \pm 1 \\ 1
\end{pmatrix}
& & 
M \begin{pmatrix}
\pm 1 \\ \pm 1 \\ 1 \\ 1 
\end{pmatrix} \mu
\end{eqnarray*}
\end{minipage}
\end{figure}
\begin{eqnarray*}
M \begin{pmatrix}
\pm 1 \\ \pm 1 \\ -1 \\ 1 \\
\end{pmatrix} =  \begin{pmatrix}
\pm 1 \\ \pm 1 \\ -1 \\ 1 \\
\end{pmatrix} & M = 
\begin{pmatrix}
1 & 0 & 0 & 0 \\
0 & 1 & 0 & 0 \\
0 & 0 & a & b \\
0 & 0 & c & d \\
\end{pmatrix}
& \\
M \begin{pmatrix}
\pm f \\ \pm f \\ -f \\ 1 \\
\end{pmatrix} =  \begin{pmatrix}
\pm f \\ \pm f \\ f \\ f \\
\end{pmatrix} & \begin{pmatrix}
a & b \\ c & d \\
\end{pmatrix}
\begin{pmatrix}
-1 \\ 1
\end{pmatrix}
=
\begin{pmatrix}
-1 \\ 1
\end{pmatrix} & a = \frac{f+1}{f-1} \text{   } b = \frac{-2f}{f-1}\\
& \begin{pmatrix}
a & b \\ c & d \\
\end{pmatrix}
\begin{pmatrix}
-f \\ 1
\end{pmatrix}
=
\begin{pmatrix}
f \\ f
\end{pmatrix} & c = -1 \text{   } d = 0
\end{eqnarray*}
\subsubsection{allgemeiner Fall}
\begin{equation*}
M = \begin{pmatrix}
\frac{2nah}{r-l} & 0 & \frac{r+l}{r-l} & 0 \\
0 & \frac{2nah}{t - h} & \frac{t+b}{t-b} & 0 \\
0 & 0 & \frac{-fern+nah}{fern-nah} & -\frac{2fern *nah}{fern - nah} \\
0 & 0 & -1 & 0 \\
\end{pmatrix} \text{    }
M^{-1} = 
\begin{pmatrix}
\frac{r-l}{2nah} & 0 & 0 & \frac{r+l}{2nah} \\
0 & \frac{t+b}{2nah} & 0 &  \frac{t+b}{2nah} \\
0 & 0 & 0 & -1 \\
0 & 0 & \frac{1}{2fern}-\frac{1}{2nah} & \frac{1}{2fern}+\frac{1}{2nah}\\
\end{pmatrix}
\end{equation*}
\subsection{Tatsächliche Koordinaten und Rasterung}
\begin{figure}[!hb]
\begin{minipage}[c]{5cm}
{\color{red} Zeichnung Raster}
\end{minipage}
\hfill
\begin{minipage}[c]{8cm}
$res_n \times res_v$ Rechteck von Pixeln \\
horizontale und vertikale Auflösung \\
hängt von der Genauigkeit ab.
$$ j= \lfloor (x+1)\frac{res_h}{2}\rfloor  = round\left(\frac{(x+1)res_n-1}{2}\right)$$
$$ i = round\left(\frac{(1-y)res_v-1}{2}\right)$$
\end{minipage}
\end{figure}
\section{Vorlenung 02.05.14}
{\color{red}Zeichnung Lichtspectrum}\\
\begin{figure}[!hb]
\begin{minipage}[c]{7.5cm}
Es gibt Licht in jeder beliebigen Wellenlänge \\
\begin{minipage}[c]{2cm}
\begin{tabular}{c}
412 nm \\
414 nm
\end{tabular}
\end{minipage}
\hfill
\begin{minipage}[r]{3cm}
 stetiger Übergang
\end{minipage}\\
Wellenlänge bestimmt die Farbwahrnehmung
\end{minipage}
\hfill
\begin{minipage}[c]{6.5cm}
natürliches Licht besteht aus verschiedenen Wellenlängen\\
{\color{red}Zeichnung  }
Sonnenlicht besitzt alle Anteile gefiltert durch Atmossphäre\\
{\color{red}  Zeichnug + monochromatisches Licht }
\end{minipage}
%\end{minipage}
\end{figure}
\begin{figure}[!hb]
\begin{minipage}[c]{6.5cm}
{\color{red}Prisma Zeichnung}
\end{minipage}
\hfill
\begin{minipage}[c]{7.5cm}
Der Raum der wahrnehmbaren Farben ist dreidimensional. Wir können nicht alle verschiedenen Mischungen von Spektralfarben unterscheiden
\end{minipage}
\end{figure}
Für das Farbsehen gibt es im Auge 3 verschiedene Arten von Zäpfchen, die unterschiedlich auf verschiedene Wellenlängen ansprechen \\
{\color{red}Zeichnung Stäbchenspektrum}
\subsection{Intensitätsverteilung I($\lambda$)}
Empfindung der S-Zäpfchen wird gegeben 
$$ \int\limits_{\lambda}\!S(\lambda) \cdot I(\lambda) \,\mathrm{d}\lambda $$
analog für M und L-Zapfen.\\
Diese drei Werde bestimmen den \underline{Farbeindruck}.
\subsection{Prinzib der Additiven Farbmischung}
\begin{figure}[!hb]
\begin{minipage}[c]{6cm}
{\color{red}Koordinatensystem}
\end{minipage}
\hfill
\begin{minipage}[c]{9cm}
verschiedene Scheinwerfer leuchten übereinander\\
\begin{minipage}[c]{4cm}
{\color{red}Zeichnung Scheinwerfer}
\end{minipage}
\hfill
\begin{minipage}[c]{4.5cm}
\vfill
\begin{itemize}
\item Bildschirm
\item Projektor
\end{itemize}
\vfill
\end{minipage}
\end{minipage}
\vfill
\begin{minipage}[c]{14cm}
\begin{minipage}[r]{7cm}
\begin{tabular}{ccccc}
R & + & G & = & Yellow \\
R & + & B & = & Magenta \\
B & + & G & = & Cyan \\
\end{tabular}
\end{minipage}
\hfill
\begin{minipage}[c]{6.8cm}
y lässt R+G durch\\
absorbiert B
\end{minipage}
\end{minipage}
\end{figure}
\subsection{Prinzip der \underline{subtraktiven} Farbmischung}
Jede durchlässige Folie lässt Licht verschiedener Wellenlängen in unterschiedlicher Intensität durch
\begin{figure}
\begin{minipage}[c]{7cm}
\begin{minipage}[c]{7cm}
{\color{red} Zeichnung graph}
\end{minipage}
\vfill
\begin{minipage}[c]{7cm}
Farbe von einer Oberfläche zurückgeworfen und genügt gleichen Gesetzen
\end{minipage}
\end{minipage}
\hfill
\begin{minipage}[c]{7cm}
\begin{minipage}[c]{7cm}
$$ I^{aus}(\lambda) = I^{ein}(\lambda) \cdot C(\lambda)$$
\end{minipage}
\vfill
\begin{minipage}[c]{7cm}
{\color{red} Zeichnung Licht}
\end{minipage}
\end{minipage}
\end{figure}
Wenn man zwei unterschiedliche Lichtquellen A und A' mit dem selben Farbeindruck mit einer dritten Lichtquelle B additiv mischt, dann erzeugen A+B und A'+B den selben Farbeindruck.\\
Wenn die Intensität proportional gesteigert wird, ändert sich nur die Helligkeit, nicht (der Farbton) die Farbe. Der Schnitt mit einer Ebene gibt schon alle Informationen über die möglichen Farbwahrnehmungen.
{\color{red}Farbdreieck}
\section{Vorlesung 14.05.14}
\subsection{Rasterung, Zeichnungen von Geraden und Kreisen der \\Bresenham-Algorithmus}

\begin{minipage}[c]{14cm}
\begin{minipage}[c]{7cm}
gegeben: zwei Endpunkte\\
$A,B \in \mathbb{Z}^2$ (Gitterebene)\\
welche Bildpunkte $(i,j) \in \mathbb{Z}^2$ soll\\ man zeichnen / einschalten?
\end{minipage}
\hfill
\begin{minipage}[c]{7cm}
\begin{minipage}[c]{7cm}
{\color{red}Gitterzeichnung}
\end{minipage}
\vfill
\begin{minipage}[c]{7cm}
Grundoperation \\
SetPixel(i,j,Farbe)\\
Farbe konstant weil vorgegeben\\
$\rightarrow$ weglassen
\end{minipage}
\end{minipage}
\end{minipage}\\
\begin{minipage}[c]{14cm}
\begin{minipage}[c]{7cm}
\begin{enumerate}
\item[1.) Annahme:] A=(0,0) (Translation)\\
B=(x,y)
\item[2.) o.B.d.A] $x,y \geq 0 \Rightarrow$ B im ersten Quadranten
\item[3.) o.B.d.A] $y \leq x \Rightarrow$ B im ersten Qktanten weil Vertauschung von x und y möglich
\end{enumerate}
\end{minipage}
\hfill
\begin{minipage}[c]{7cm}
{\color{red}Halbkreiszeichnung}
\end{minipage}
\end{minipage}\\
\begin{minipage}[c]{14cm}
\begin{minipage}[r]{7cm}
$$\text{Steigung } m=\frac{y}{x} \;\;\; 0\leq \frac{y}{x} \leq $$
\end{minipage}
\hfill
\begin{minipage}[c]{7cm}
{\color{red} gitterliniennetz}
\end{minipage}
\end{minipage}\\
Festlegung : wir zeichnen für jede vertikale Gitterlinie $i=0,1,…,x$ genau ein Pixel(i,j)
\begin{minipage}[c]{14cm}
\begin{minipage}[c]{7cm}
\begin{program}
\FOR i=0,1,2,...,x
	|SetPixel|(i,round \frac{y}{x})
\end{program}
\end{minipage}
\hfill
\begin{minipage}[c]{7cm}
Wie löscht man zweideutigkeiten beim Runden auf ?\\
konsistente Regel\\
z.B.: immer zum kleineren Koordinatenwort rund
\end{minipage}
\end{minipage}\\
Nachteil: Multiplikation mit gebrochenen Zahlen
\end{document}