\documentclass[11pt]{article}

\usepackage[utf8]{inputenc}
\usepackage{geometry}
\usepackage{color,graphicx,framed}
\usepackage{amsmath,amsfonts,amssymb}
\usepackage{listings}
\usepackage[section]{placeins}
\usepackage{pstricks,pst-tree,pst-node}
\usepackage{dsfont}
\usepackage{ulem}


\title{Computergrafik - Übungsblatt 01}
\author{Björn Rathjen,\\Patrick Winterstein}
\date{24.10.14}

\begin{document}
\maketitle
\newpage
\section{(0 Punkte)}
\begin{enumerate}
\item[(a)]Berechnen Sie eine Rotation
$$ \text{R: } \begin{pmatrix}
x \\ y \\ 1
\end{pmatrix} \rightarrow \begin{pmatrix}
m_{11} & m_{12} & m_{13} \\
m_{21} & m_{22} & m_{23} \\
0 & 0 & 1 \\
\end{pmatrix} \begin{pmatrix}
x \\ y \\ 1 \\
\end{pmatrix}, $$
die die beiden Punkte (mit karthesischen Koordinaten) (0,0) und (5,0) auf die Punkte (2,1) und (5,5) abbildet.\\
Zur Berechnung der Rotation  
\begin{eqnarray*}
\begin{pmatrix}
m_{11} & m_{12} & m_{13} \\
m_{21} & m_{22} & m_{23} \\
0 & 0 & 1 \\
\end{pmatrix}
\times
\begin{pmatrix}
0 \\ 0 \\ 1 \\
\end{pmatrix}
&=&
\begin{pmatrix}
2 \\ 1 \\ 1 \\
\end{pmatrix}
\end{eqnarray*}
resultierendes Gleichungssystem : \\
1
\begin{eqnarray*}
m_{11} \cdot 0 + m_{12} \cdot 0 + m_{13} \cdot 1  &=& 2 \\
m_{21} \cdot 0 + m_{22} \cdot 0 + m_{23} \cdot 1  &=& 1 \\
0 \cdot 0 + 0 \cdot 0 + 1 \cdot 1 &=& 1 \\ 
\end{eqnarray*}
2
\begin{eqnarray*}
m_{13} &=& 2 \\
m_{23} &=& 1 \\
1 &=& 1 \\
\end{eqnarray*}
resultierende Matrix:\\
3
$$\begin{pmatrix}
m_{11} & m_{12} & 2 \\
m_{21} & m_{22} & 1 \\
0 & 0 & 1 \\
\end{pmatrix}$$
4
\begin{eqnarray*}
\begin{pmatrix}
m_{11} & m_{12} & 2 \\
m_{21} & m_{22} & 1 \\
0 & 0 & 1 \\
\end{pmatrix}
\times
\begin{pmatrix}
5 \\ 0 \\ 1 \\
\end{pmatrix}
&=&
\begin{pmatrix}
5 \\ 5 \\ 1 \\
\end{pmatrix}
\end{eqnarray*}
5
\begin{eqnarray*}
m_{11} \cdot 5 + m_{12} \cdot 0 + 2 \cdot 1  &=& 5 \\
m_{21} \cdot 5 + m_{22} \cdot 0 + 1 \cdot 1 &=& 5 \\
0 \cdot 0 + 0 \cdot 0 + 1 \cdot 1 &=& 1 \\ 
\end{eqnarray*}
6
\begin{eqnarray*}
m_{11} \cdot 5 &=& 3 \\
m_{21} \cdot 5 &=& 4 \\
1 &=& 1 \\ 
\end{eqnarray*}
7
\begin{eqnarray*}
m_{11} &=& \frac{3}{5} \\
m_{21} &=& \frac{4}{5} \\
\end{eqnarray*}
8\\
resultierende Matrix:
$$\begin{pmatrix}
\frac{3}{5} & m_{12} & 2 \\
\frac{4}{5} & m_{22} & 1 \\
0 & 0 & 1 \\
\end{pmatrix}$$
9\\
da Rotation $\rightarrow $
$$\begin{pmatrix}
\frac{3}{5} & -\frac{4}{5} & 2 \\
\frac{4}{5} & \frac{3}{5} & 1 \\
0 & 0 & 1 \\
\end{pmatrix}$$
\item[(b)] Bestimmten Sie den Punkt $z$ der Ebene mit $z = R(z)$ (den Fixpunkt; den Punkt, um den gedreht wird).
\begin{eqnarray*}
\begin{pmatrix}
\frac{3}{5} & -\frac{4}{5} & 2 \\
\frac{4}{5} & \frac{3}{5} & 1 \\
0 & 0 & 1 \\
\end{pmatrix}
\times 
\begin{pmatrix}
x \\ y \\ 1 \\
\end{pmatrix}
&=&
\begin{pmatrix}
x \\ y \\ 1 \\
\end{pmatrix} \\
\end{eqnarray*}
resultierendes Gleichungssystems
\begin{eqnarray*}
\frac{3}{5} \cdot x + -\frac{4}{5} \cdot y + 2 &=& x \\
\frac{4}{5} \cdot x + \frac{3}{5} \cdot y + 1 &=& y \\ 
\end{eqnarray*}
\begin{eqnarray*}
3 \cdot x - 4 \cdot y + 10 &=& 5 \cdot x \\
4 \cdot x + 3 \cdot y + 5 &=& 5 \cdot y \\ 
\end{eqnarray*}
\begin{eqnarray*}
2 \cdot x &=& - 4 \cdot y + 10\\
4 \cdot x + 5 &=& 2 \cdot y \\ 
\end{eqnarray*}
2 - 2 * 1
\begin{eqnarray*}
5 &=& 10 \cdot y - 20\\
25 &=& 10 \cdot y \\
2,5 &=& y\\
\\
x &=& -2 \cdot 2,5 + 5\\
x &=& 0 \\
\end{eqnarray*}
Punkt ist (0 / 2,5).
\item[(c)] Um welchen Winkel wird die Ebene dabei gedreht? (in Uhrzeigersinn bzw. gegen den Uhrzeigersinn?)
$$ \cos \alpha = \frac{3}{5} $$
$$ \alpha = 53,13$$
\end{enumerate}
\section{Freiheitsgrade (0 Punkte)}
Wieviele Paare von Urbildpunkten und Bildpunkten muss man im Allgemeinen festlegen (beliebig, fast beliebig, mit Einschränkungen), um die folgenden Abbildungsarten im $\mathbb{R^2}$ bzw. $ \mathbb{R}^3$ eindeutig zu charakterisieren? Geben Sie kurze Begründungen.
\begin{enumerate}
\item[(a)] Isometrie (starre Bewegung)
\item[(b)] Affine Abbildung
\end{enumerate}
Anmerkung: Manche Information kann man auch über ein einzelnes Bit speichern, an-
statt ein zusätzliches Punktepaar zu verwenden. Diese Fälle sollen erkannt werden.
\section{(0 Punkte)} 
Schreiben Sie die Transformationsmatrix M , die der Nacheinanderausführung der folgenden Transformationen (in dieser Reihenfolge) entspricht:
\begin{enumerate}
\item[(a)] Eine Rotation um den Ursprung um 90 Grad nach links.
\item[(b)] Eine Streckung der x-Achse um den Faktor 2. (Die y-Achse bleibt unverändert.)
\item[(c)] Eine Translation um den Vektor (2, 1).
\item[(d)] Eine Rotation um den Ursprung um 90 Grad nach links.
\end{enumerate}
Auf welche Punkte werden die drei Punkte (4, 2), (3, 3), (3, 7) am Ende abgebildet?
\section{(0 Punkte)}
\begin{enumerate}
\item[(a)] Welche geometrischen Transformation wird durch die Abbildung
$$ \text{A: } x \rightarrow \begin{pmatrix}
1&0\\ 
0&-1\\
\end{pmatrix}x$$
beschrieben?\\
Spiegelung an der x-Achse, weil :
$$ \text{A: } x \rightarrow \begin{pmatrix}
1&0\\ 
0&-1\\
\end{pmatrix}
\times
\begin{pmatrix}
x \\ y \\
\end{pmatrix} = \begin{pmatrix}
x \\ -y\\
\end{pmatrix} $$
\item[(b)] Sei R eine Rotation um 90 Grad nach rechts um den Ursprung. Wenden Sie folgende drei Transformationen in der folgenden Reihenfolge an:
$$R,A,R^{-1}$$
Bestimmen Sie die Matrix M , welche der Verknüpfung der drei Abbildungen entspricht.\\
$$
R =
\begin{pmatrix}
\cos 90 & -\sin 90 \\
\sin 90 & \cos 90 \\
\end{pmatrix}\\
$$
$$ 
\begin{pmatrix}
\cos 90 & -\sin 90 \\
\sin 90 & \cos 90 \\
\end{pmatrix}
\times
\begin{pmatrix}
1 & 0 \\
0 & -1 \\
\end{pmatrix}
\times
\begin{pmatrix}
\cos -90 & -\sin -90 \\
\sin -90 & \cos -90 \\
\end{pmatrix}
$$
$$ 
\begin{pmatrix}
\cos 90 & -\sin 90 \\
\sin 90 & \cos 90 \\
\end{pmatrix}
\times
\begin{pmatrix}
1 & 0 \\
0 & -1 \\
\end{pmatrix}
\times
\begin{pmatrix}
\cos 90 & \sin 90 \\
-\sin 90 & \cos 90 \\
\end{pmatrix}
$$
$$ 
\begin{pmatrix}
0 & -1 \\
1 & 0 \\
\end{pmatrix}
\times
\begin{pmatrix}
1 & 0 \\
0 & -1 \\
\end{pmatrix}
\times
\begin{pmatrix}
0 & 1 \\
-1 & 0 \\
\end{pmatrix}
$$
$$
\begin{pmatrix}
0 & 1 \\
1 & 0 \\
\end{pmatrix}
\times
\begin{pmatrix}
0 & 1 \\
-1 & 0 \\
\end{pmatrix}
$$
$$
\begin{pmatrix}
-1 & 0 \\
0 & 1 \\
\end{pmatrix}
$$

\item[(c)] Bestimmen Sie die Fixpunkte von M (die Punkte x mit $M x = x$).
\begin{eqnarray*}
\begin{pmatrix}
-1 & 0 \\
0 & 1 \\
\end{pmatrix} 
\times
\begin{pmatrix}
x \\ y
\end{pmatrix}
&=& 
\begin{pmatrix}
x \\ y
\end{pmatrix}
\\
-x &=& x \\
y &=& y \\
\end{eqnarray*}
\item[(d)] Welche geometrische Transformation wird durch M beschrieben?\\
Spiegelung an der y-Achse.
\end{enumerate}
\section{(0 Punkte)}
Wenden Sie die projektive Transformation $x \rightarrow Mx$ mit
$$
\begin{pmatrix}
2 & 3 & 4\\
-1 & 0 & 0 \\
2 & 2 & 1 \\
\end{pmatrix}
$$
auf die Quadrate eines Schachbrettmusters.\\
$\{ (x, y) \; \in \; \mathbb{R}^{2} \;  \exists \; i, j \in \mathbb{Z}, 0 \leq i \leq x \leq i + 1 \leq 8, 0 \leq j \leq y \leq j + 1 \leq 8, i + j \text{ ungerade} \}$
an und zeichnen Sie das Ergebnis.\\

\begin{tabular}{ccccccccc}
$
\begin{pmatrix}
4\\
0\\
1\\
\end{pmatrix}
$
&
$
\begin{pmatrix}
6\\
0\\
1\\
\end{pmatrix}
$
\end{tabular}

\section{(0 Punkte)}
\begin{enumerate}
\item[(a)] Berechnen Sie den Schnittpunkt P der beiden Geraden
$$3x + 4y = 5$$
$$4x + 5y = -6$$
in \textit{homogenen} Koordinaten.
\item[(b)] Bestimmen Sie die Gleichung der Geraden g durch P und den Punkt
$$
\begin{pmatrix}
3 \\ 2 \\ 6\\
\end{pmatrix}
$$
\item[(c)] Schneiden Sie $g$ mit der Ferngeraden.
\end{enumerate}
\section{Rotation um eine beliebige Achse (10 Punkte)}
Bestimmen Sie die Transformationsmatrix für eine Rotation um die Gerade durch den
Ursprung in Richtung des Vektors
$$
\begin{pmatrix}
-4 \\ 2 \\ 3 \\
\end{pmatrix}
$$
um einen Winkel von 30 Grad gegen den Uhrzeigersinn, wenn man vom Ursprung aus in
Richtung von u schaut. Verwenden Sie dazu eine Methode Ihrer Wahl, z.B. die folgende:
\begin{enumerate}
\item[1.] Drehe den Vektor u in die yz-Ebene (z. B. um die z-Achse).
$$
\begin{pmatrix}
3/ \sqrt{29} & -\sqrt{20}/\sqrt{29} & 0 \\
\sqrt{20}/\sqrt{29} & 3/ \sqrt{29} & 0 \\
0 & 0 & 1 \\
\end{pmatrix}
$$
\item[2.] Drehe den Vektor weiter in die z-Achse (um die x-Achse).
$$
\begin{pmatrix}
1 & 0 & 0 \\
0 & \sqrt{20}/\sqrt{29} & -3/ \sqrt{29}\\
0 & 3/ \sqrt{29} & \sqrt{20}/\sqrt{29} \\
\end{pmatrix}
$$
\item[3.] Drehe um 30 Grad um die z-Achse.
$$
\begin{pmatrix}
\cos 30 & -\sin 30 & 0 \\
\sin 30 & \cos 30 & 0 \\
0 & 0 & 1 \\
\end{pmatrix}
$$
\item[4.] Führe die Transformationen aus Schritt 2 und 1 rückwärts aus. Kontrollieren Sie, dass der Punkt u auf sich selbst abgebildet wird.
$$
\begin{pmatrix}
1 & 0 & 0 \\
0 & \sqrt{20}/\sqrt{29} & 3/ \sqrt{29}\\
0 & -3/ \sqrt{29} & \sqrt{20}/\sqrt{29} \\
\end{pmatrix}
$$

\end{enumerate}
gesamt : 
$$
\begin{pmatrix}
1 & 0 & 0 \\
0 & \sqrt{20}/\sqrt{29} & -3/ \sqrt{29}\\
0 & 3/ \sqrt{29} & \sqrt{20}/\sqrt{29} \\
\end{pmatrix}
\times
\begin{pmatrix}
3/ \sqrt{29} & -\sqrt{20}/\sqrt{29} & 0 \\
\sqrt{20}/\sqrt{29} & 3/ \sqrt{29} & 0 \\
0 & 0 & 1 \\
\end{pmatrix}
=
\begin{pmatrix}
3/ \sqrt{29} & -\sqrt{20}/\sqrt{29} & 0 \\
20 / 29 & 3 \cdot \sqrt{20}/29 & -3/ \sqrt{29} \\
3 \cdot \sqrt{20}/29 & 9 /29 & \sqrt{20}/\sqrt{29} \\
\end{pmatrix}
$$
$$
\begin{pmatrix}
3/ \sqrt{29} & -\sqrt{20}/\sqrt{29} & 0 \\
20 / 29 & 3 \cdot \sqrt{20}/29 & -3/ \sqrt{29} \\
3 \cdot \sqrt{20}/29 & 9 /29 & \sqrt{20}/\sqrt{29} \\
\end{pmatrix}
\times
\begin{pmatrix}
1 & 0 & 0 \\
0 & \sqrt{20}/\sqrt{29} & -3/ \sqrt{29}\\
0 & 3/ \sqrt{29} & \sqrt{20}/\sqrt{29} \\
\end{pmatrix}
=
\begin{pmatrix}
 & & 
\end{pmatrix}
$$


\section{Arithmetische Komplexität (10 Punkte)}
Betrachten Sie das Verfahren der vorherigen Aufgabe für einen allgemeinen Vektor u und bestimmen Sie die Anzahl der arithmetischen Operationen (Addition und Subtraktion, Multiplikation, Division, Quadratwurzel).\\
5 mal Multiplitkation von Vektor mit einer Matrix.\\
das sind 5 mal 3 mal (3 mult 3 add)\\
um matrix zu berechnen 4 mal (4 divisionen + 2 mal (2add 3mult sqrt) + 2 mal (udd 2mult sqrt)
\begin{description}
\item[Additionen / Subtraktionen] 45 + 24 = 69
\item[Multiplikationen] 45 + 40
\item[Divisionen]16
\item[Quadratwurzeln] 16
\end{description}
\section{Projektives Bild einer Strecke (10 Punkte)}
\begin{enumerate}
\item[(a)] Zerlegen Sie die Strecke von (-1, 0) bis (0, 1) in 10 gleiche Teile und wenden Sie auf die Zwischenpunkte die projektive Transformation $x \rightarrow M x$ mit\\
$
\begin{pmatrix}
2 & 3 &  4 \\
-1 & 0 & 0 \\
2 & 2 & 1 \\
\end{pmatrix}
$\\


aus Aufgabe 5 an. Zeichnen Sie das Ergebnis. Was ist das Bild der ganzen Strecke
unter dieser Transformation?
\begin{eqnarray*}
\begin{pmatrix}
2 & 3 &  4 \\
-1 & 0 & 0 \\
2 & 2 & 1 \\
\end{pmatrix}
\times
\begin{pmatrix}
-1 \\ 0 \\ 1 \\
\end{pmatrix}
&=&
\begin{pmatrix}
-2 + 4 \\
1 \\
-2 + 1 \\
\end{pmatrix}
=
\begin{pmatrix}
2 \\ 1 \\ -1 \\
\end{pmatrix}
=
\begin{pmatrix}
-2 \\ -1 \\ 1 \\
\end{pmatrix}
\\
\begin{pmatrix}
-0,9 \\ 0,1 \\ 1 \\
\end{pmatrix}
\rightarrow
\begin{pmatrix}
2,5 \\ 0,9 \\ -0,6 \\
\end{pmatrix}
&=&
\begin{pmatrix}
-25/6 \\ -3/2 \\ 1 \\
\end{pmatrix}
= 
\begin{pmatrix}
-4,16 \\ -1,5 \\ 1
\end{pmatrix}\\
\begin{pmatrix}
-0,8 \\ 0,2 \\ 1 \\
\end{pmatrix}
\rightarrow
\begin{pmatrix}
3 \\ 0,8 \\ -0,2 \\
\end{pmatrix}
&=&
\begin{pmatrix}
-15 \\ -4 \\ 1 \\
\end{pmatrix}\\
\begin{pmatrix}
-0,7 \\ 0,3 \\ 1 \\
\end{pmatrix}
\rightarrow
\begin{pmatrix}
3,5 \\ 0,7 \\ 0,2 \\
\end{pmatrix}
&=&
\begin{pmatrix}
35/2 \\ 7/2 \\ 1 \\
\end{pmatrix}
=
\begin{pmatrix}
17,5 \\ 3,5 \\ 1
\end{pmatrix}\\
\begin{pmatrix}
-0,6 \\ 0,4 \\ 1 \\
\end{pmatrix}
\rightarrow
\begin{pmatrix}
4 \\ 0,6 \\ 0,6 \\
\end{pmatrix}
&=&
\begin{pmatrix}
20/3 \\ 1 \\ 1 \\
\end{pmatrix}
=
\begin{pmatrix}
6,66 \\ 1 \\ 1
\end{pmatrix}\\
\begin{pmatrix}
-0,5 \\ 0,5 \\ 1 \\
\end{pmatrix}
\rightarrow
\begin{pmatrix}
4,5 \\ 0,5 \\ 1 \\
\end{pmatrix}
&=&
\begin{pmatrix}
4,5 \\ 0,5 \\ 1 \\
\end{pmatrix}\\\\
\begin{pmatrix}
-0,4 \\ 0,6 \\ 1 \\
\end{pmatrix}
\rightarrow
\begin{pmatrix}
5 \\ 0,4 \\ 1,4 \\
\end{pmatrix}
&=&
\begin{pmatrix}
25/7 \\ 2/7 \\ 1 \\
\end{pmatrix}
=
\begin{pmatrix}
3,57 \\ 0,285 \\ 1 \\
\end{pmatrix}
\\
\begin{pmatrix}
-0,3 \\ 0,7 \\ 1 \\
\end{pmatrix}
\rightarrow
\begin{pmatrix}
5,5 \\ 0,3 \\ 1,8 \\
\end{pmatrix}
&=&
\begin{pmatrix}
55/18 \\ 1/6 \\ 1 \\
\end{pmatrix}
=
\begin{pmatrix}
3,05 \\ 0,16 \\ 1 \\ 
\end{pmatrix}
\\
\begin{pmatrix}
-0,2 \\ 0,8 \\ 1 \\
\end{pmatrix}
\rightarrow
\begin{pmatrix}
6 \\ 0,2 \\ 2,2 \\
\end{pmatrix}
&=&
\begin{pmatrix}
30/11 \\ 1/11 \\ 1 \\
\end{pmatrix}
=
\begin{pmatrix}
2,72 \\ 0,099 \\ 1
\end{pmatrix}
\\
\begin{pmatrix}
-0,1 \\ 0,9 \\ 1 \\
\end{pmatrix}
\rightarrow
\begin{pmatrix}
6,5 \\ 0,2 \\ 2,6 \\
\end{pmatrix}
&=&
\begin{pmatrix}
\frac{65}{26} \\ \frac{1}{13} \\ 1 \\
\end{pmatrix}
=
\begin{pmatrix}
2,05 \\ 0,076 \\ 1
\end{pmatrix}
\\
\begin{pmatrix}
0 \\ 1 \\ 1 \\
\end{pmatrix}
\rightarrow
\begin{pmatrix}
7 \\ 0 \\ 3 \\
\end{pmatrix}
&=&
\begin{pmatrix}
\frac{7}{5} \\ 0 \\ 1 \\
\end{pmatrix}
=
\begin{pmatrix}
1,4 \\ 0 \\ 1
\end{pmatrix}\\
\end{eqnarray*} 
\item[(b)] Charakterisieren Sie diejenigen Strecken, deren Bild unter einer gegebenen projektiven Transformation wieder eine Strecke ist. (Also nicht ein unendlicher Strahl oder etwas anderes.)\\
Es existiert keine Strecke zwischen den Punkten 3 und 4 sonst schon.
$$
\begin{pmatrix}
2 & 3 &  4 \\
-1 & 0 & 0 \\
2 & 2 & 1 \\
\end{pmatrix}
\times
\begin{pmatrix}
x \\ y \\ z
\end{pmatrix} 
= 
\begin{pmatrix}
2 \cdot x + 3 \cdot y + 4 \\
-x\\
2 \cdot x + 2 \cdot y + 1 \\
\end{pmatrix}
$$
Da y immer $x + 1 \Rightarrow $
$$
\begin{pmatrix}
2 \cdot x + 2 \cdot (x + 1) + 4\\
-x \\
2 \cdot x + 2 \cdot (x + 1) + 1\\
\end{pmatrix}
=
\begin{pmatrix}
4 \cdot x + 6 \\
-x\\
4 \cdot x + 3 \\
\end{pmatrix}
$$
da das resultierende z nicht 0 werden darf, ist bei $x = -3/4$ ein Fernpunkt und Punktpaare, zwischen denen dieser Punkt liegt, sind nach der Transformation nicht miteinander verbunden.
\end{enumerate}
\end{document}

