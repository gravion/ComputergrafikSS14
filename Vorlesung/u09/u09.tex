\documentclass[11pt]{article}

\usepackage[utf8]{inputenc}
\usepackage{geometry}
\usepackage{color,graphicx,framed}
\usepackage{amsmath,amsfonts,amssymb}
\usepackage{listings}
\usepackage[section]{placeins}
\usepackage{pstricks,pst-tree,pst-node}
\usepackage{dsfont}
\usepackage{ulem}
\usepackage{nicefrac}

\title{Computergrafik\\Übungsblatt 09}
\author{Björn Rathjen}
\date{SS14}

\begin{document}
\maketitle
\newpage
\section{Aufgabe 46 : Verstädnisfrage}
Gegenüberstellung von Kegelmantel als
\begin{itemize}
\item Dreiecksgitter
\item Gleichung $x^2 + y^2 = z^2 \; , \; 0\leq z \leq 1$ 
\end{itemize}
Die Punkte beschreiben die Trennung von Geometrie und Kombinatorik\\

\begin{tabular}{c|c|c}
Operation / Aufgabenstellung &
Dreiecksgitter &
Gleichung \\ \hline
 
Darstellung & 
Annäherung an die Form &
nummerisch genaue Darstellung \\ \hline 

Beleuchtung / Färbung & 
\begin{minipage}[c]{5cm} performanter aber nicht exact \end{minipage} &
\begin{minipage}[c]{5cm} sehr aufwändig da keine lineare Interpolation möglich, genau \end{minipage} \hspace{0.3cm}\\ \hline  

Algorithmen &
\hspace{0.3cm }\begin{minipage}[c]{5cm} Gouraud-Schattierung, Phong-Schattierung– Phong-Blinn-Schattierung \end{minipage}& 
Bresenham (Raserung -> Aliasing),\\ \hline \hspace{0.3cm}

Datenstruktur &
darstellbar &
nur Formel somit immer eine neue Berechnung notwendig \\

\end{tabular}\\
Raytracing … forward raytracing … rayshooting 
Datenstrucktur Durchlaufbarkeit
Supersampling \\

\end{document}